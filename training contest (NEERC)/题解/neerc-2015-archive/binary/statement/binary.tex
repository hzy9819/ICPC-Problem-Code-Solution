\begin{problem}{Binary vs Decimal}{binary.in}{binary.out}{2 seconds}

% Author: Mikhail Tikhomirov (idea & text)

Bruce has recently got a job at NEERC (Numeric Expression Engineering \& Research Center) facility, which studies and produces
many kinds of curious numbers. His first assignment is to perform a study of bindecimal numbers.

A positive integer is called \emph{bindecimal} if its decimal representation is a suffix of its binary representation; both binary and decimal representations are
considered without leading zeros.
For example, $10_{10} = 10\mathbf{10}_2$, thus, $10$ is a bindecimal number. The numbers $1010_{10} = 111111\mathbf{0010}_2$ and $42_{10} = 1010\mathbf{10}_{10}$
are, evidently, not bindecimal.

First of all, Bruce is going to create a list of bindecimal numbers. Help him find the $n$-th smallest bindecimal number.

\InputFile

The first and the only line contains one integer~--- $n$ ($1 \leq n \leq 10\,000$).

\OutputFile

Print one integer~--- the $n$-th smallest bindecimal number in decimal notation.

\Example

\begin{example}
\exmp{
1
}{
1
}%
\exmp{
2
}{
10
}%
\exmp{
10
}{
1100
}%
\end{example}

\Note

Here is a table with the first few numbers which contain only 0's and 1's in their decimal representation (it is clear that all other numbers are not bindecimal):

\begin{center}

\begin{tabular}{|r|r|l|}
\hline
Decimal & Binary & Comment \\
\hline
1 & {\bf 1} & 1st bindecimal number \\
\hline
10 & 10{\bf 10} & 2nd bindecimal number \\
\hline
11 & 10{\bf 11} & 3rd bindecimal number \\
\hline
100 & 1100{\bf 100} & 4th bindecimal number \\
\hline
101 & 1100{\bf 101} & 5th bindecimal number \\
\hline
110 & 1101{\bf 110} & 6th bindecimal number \\
\hline
111 & 1101{\bf 111} & 7th bindecimal number \\
\hline
1000 & 111110{\bf 1000} & 8th bindecimal number \\
\hline
1001 & 111110{\bf 1001} & 9th bindecimal number \\
\hline
1010 & 111111{\bf 0010} & Not a bindecimal number \\
\hline
1011 & 111111{\bf 0011} & Not a bindecimal number \\
\hline
1100 & 1000100{\bf 1100} & 10th bindecimal number \\
\hline
\end{tabular}

\end{center}

\end{problem}
